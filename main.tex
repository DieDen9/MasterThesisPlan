%%%%%%%%%%%%%%%%%%%%%%%%%%%%%%%%%%%%%%%%%%%%%%%%%%%%%%%%%%%%%%%%%%%%%%
% LaTeX Example: Project Report
%
% Source: http://www.howtotex.com
%
% Feel free to distribute this example, but please keep the referral
% to howtotex.com
% Date: March 2011 
% 
%%%%%%%%%%%%%%%%%%%%%%%%%%%%%%%%%%%%%%%%%%%%%%%%%%%%%%%%%%%%%%%%%%%%%%
% How to use writeLaTeX: 
%
% You edit the source code here on the left, and the preview on the
% right shows you the result within a few seconds.
%
% Bookmark this page and share the URL with your co-authors. They can
% edit at the same time!
%
% You can upload figures, bibliographies, custom classes and
% styles using the files menu.
%
% If you're new to LaTeX, the wikibook is a great place to start:
% http://en.wikibooks.org/wiki/LaTeX
%
%%%%%%%%%%%%%%%%%%%%%%%%%%%%%%%%%%%%%%%%%%%%%%%%%%%%%%%%%%%%%%%%%%%%%%
% Edit the title below to update the display in My Documents
%\title{Project Report}
%
%%% Preamble
\documentclass[paper=a4, fontsize=10pt,margin=0.2in]{scrartcl}
\usepackage[T1]{fontenc}
\usepackage{fourier}
\usepackage[utf8]{inputenc}
\usepackage[english]{babel}															% English language/hyphenation
\usepackage[protrusion=true,expansion=true]{microtype}	
\usepackage{amsmath,amsfonts,amsthm} % Math packages
\usepackage[pdftex]{graphicx}	
\usepackage{url}
\usepackage{pgfgantt}
\usepackage[most]{tcolorbox}
\usepackage{enumitem}
 \pagenumbering{gobble}
\usepackage[compact]{titlesec}
\titlespacing{\subsection}{0pt}{*0}{*0}



%%% for SWOT
\colorlet{helpful}{lime!70}
\colorlet{harmful}{red!30}
\colorlet{internal}{yellow!20}
\colorlet{external}{cyan!30}
\colorlet{S}{helpful!50!internal}
\colorlet{W}{harmful!50!internal}
\colorlet{O}{helpful!50!external}
\colorlet{T}{harmful!50!external}
\colorlet{SO}{S!50!O}
\colorlet{WO}{W!50!O}
\colorlet{ST}{S!50!T}
\colorlet{WT}{W!50!T}

\tcbset{
    mybox/.style 2 args={%
        enhanced,
%       blankest,
        sharp corners, notitle,
        before skip=6pt, after skip=6pt,
        watermark text=#1, frame hidden,
        colback=#2,
    },
    mybox/.default={mybox={A}{white}},
}


\newenvironment{myitemize}{%
    \begin{itemize}[leftmargin=0pt] \footnotesize }%
    {\end{itemize}}
    
%%% Gantt related definitions

\definecolor{barblue}{RGB}{153,204,254}
\definecolor{groupblue}{RGB}{51,102,254}
\definecolor{linkred}{RGB}{165,0,33}
\renewcommand\sfdefault{phv}
\renewcommand\mddefault{mc}
\renewcommand\bfdefault{bc}
\setganttlinklabel{s-s}{START-TO-START}
\setganttlinklabel{f-s}{FINISH-TO-START}
\setganttlinklabel{f-f}{FINISH-TO-FINISH}
\sffamily

%%% Custom sectioning
\usepackage{sectsty}
\allsectionsfont{\centering \normalfont\scshape}

%%% Custom headers/footers (fancyhdr package)
\usepackage{fancyhdr}
\pagestyle{fancyplain}
\fancyhead{}											% No page header
\fancyfoot[L]{}											% Empty 
\fancyfoot[C]{}											% Empty
\fancyfoot[R]{\thepage}									% Pagenumbering
\renewcommand{\headrulewidth}{0pt}			% Remove header underlines
\renewcommand{\footrulewidth}{0pt}				% Remove footer underlines
\setlength{\headheight}{13.2pt}


%%% Equation and float numbering
\numberwithin{equation}{section}		% Equationnumbering: section.eq#
\numberwithin{figure}{section}			% Figurenumbering: section.fig#
\numberwithin{table}{section}				% Tablenumbering: section.tab#


%%% Maketitle metadata
\newcommand{\horrule}[1]{\rule{\linewidth}{#1}} 	% Horizontal rule

\title{
		\vspace{1in} 	
		\usefont{OT1}{bch}{b}{n}
		\horrule{0.5pt} \\[0.4cm]
		\huge  Master Thesis Project Plan\\ [0.4cm]
        \normalfont \large \textsc{Detecting Transport Hubs Using TensorFlow} \\ [20pt]
		\horrule{0.5pt} \\[10pt]
        \
}


\author{
		Author: \\
		\normalfont \normalsize Mahmoud Sukkar \\ [2cm] Supervisor: \\ \normalfont \normalsize Shafiq Ur Rehman \\ [2cm]}     		 
\date{\today}


%%% Begin document
\begin{document}

\maketitle
\pagebreak
\tableofcontents
\newpage

\section{Introduction}
This project plan is written towards the fulfillment of Master Thesis in Electronics for Robot and Control master program at Umeå University.
The thesis holds the title "Detecting Transport Hubs using TensorFlow" and held at Scania. This plan givens an overview of project description, problem definition and research methods along with planned activities in order to meet the requirement of the project.
\subsection{Project Overview}
Transportation has become a vital part in people's ordinary life. Looking at it from the single perspective of connecting suppliers with consumers, we can realize partially how big and complex the transportation network is. In addition, it will be beneficial if we could find a pattern or usages for different nodes within the transportation network. Those interesting nodes are the transport hubs where vehicles stop for some purposes. It could be a fuel station, supermarket or even cargo terminal.  \\ \\
Scania has more than 300 thousand connected vehicles that keep sending location data and it is our concern to analyze the data and extract good information that helps in identifying potential transport hubs. A good classification of transport hub may infer the usage of the vehicle that uses it since vehicle movements will be between similar or related transport hubs.     

\subsection{Purpose and goal}
The goal behind detecting transport hubs is to get a deeper knowledge of how Scania trucks are being used which can make it easier to optimize transport flows and developing a better understanding of costumer's needs and preferences.
\subsection{Deliverables}
Trained transport hub classifier using machine learning algorithms to be delivered and evaluated against ground truth data. \\ 

\section{Plan for the project}
In this section, project plan is presented during different phases of the project, followed by a calendar that illustrated activities and milestones to achieve.
\subsection{Before start}
\begin{itemize}
\item A pre-study to be concluded in this phases to get an overview of solutions followed to solve the classification problem.
\item Reading documentations of frameworks that will be used in this Thesis such as HopsWork and TensorFlow.
\item Implementations of small examples will be done along the pre-study to grasp a better understanding of the problem.
\item Getting familiar with the provided data and discover the most relevant data to the problem.
\end{itemize}
\subsection{During the project}
Following tasks to be held in order during the thesis project:
\begin{itemize}
\item Extract good and relevant dataset from the database.
\item Construct a binary classifier model for simplified problem statement e.g. (Fuel related transport hubs).
\item Iteratively testing and evaluating the binary classifier.
\item Scaling up the classifier to include more classes.
\item Iteratively testing and evaluating the classifier.
\item Acceptance of project by all supervisors and examiners. 

\item Construct a predictive model for truck usage if time permits.
\end{itemize}
It should be noted also that thesis report writing and regular meeting with my supervisor shall take a place.

\subsection{After the project}
After all deliverables and requirements are met, following activities will be done:
\begin{itemize}
\item Evaluating results.
\item Writing Discussion/conclusion part of the thesis.
\item finalizing and reviewing the thesis report.
\item Preparing presentation for thesis defense.
\item Master Thesis defense.
\end{itemize}

\subsection{Project plan schedule}
The following chart represents preliminary Thesis plan starting from week 3 until week 24. Milestones in project are shown in red font. The TW label refers to Thesis writing. It should be noted that each testing and evaluating phase is iterative. Also, Plan will be modified upon regular meetings with Thesis supervisor.  \\
% gantt chart for the project plan
\begin{flushleft}
\begin{ganttchart}[
 canvas/.append style={fill=none, draw=black!5, line width=.75pt},
    hgrid style/.style={draw=black!5, line width=.75pt},
    vgrid={*1{draw=black!5, line width=.75pt}},
    today=3,
    today rule/.style={
      draw=black!64,
      dash pattern=on 3.5pt off 4.5pt,
      line width=1.5pt
    },
    y unit title=0.6cm,
    y unit chart=0.6cm,
    today label font=\small\bfseries,
    title/.style={draw=none, fill=none},
    title label font=\bfseries\footnotesize,
    title label node/.append style={below=7pt},
    include title in canvas=true,
    bar label font=\mdseries\footnotesize \color{gray!70},
    bar label node/.append style={left=0.1cm},
    bar/.append style={draw=none, fill=red!63},
    bar incomplete/.append style={fill=barblue},
    bar progress label font=\mdseries\footnotesize\color{black!70},
    group/.append style={draw=none, fill=green!63},
    group height=.3,
    group peaks height=.2,
    group incomplete/.append style={fill=groupblue},
    group left shift=0,
    group right shift=0,
    group height=.2,
    group peaks tip position=0,
    group label node/.append style={left=.6cm},
    group progress label font=\bfseries\small,
    milestone label font=\footnotesize \color{red},
    link/.style={-latex, line width=1.5pt, linkred},
    link label font=\tiny \bfseries,
    link label node/.append style={below left=-2pt and 0pt}    
]{1}{24}
\footnotesize

  \gantttitle{Thesis Project Plan}{21} \\
  \gantttitlelist{3,...,24}{1} \\
  \ganttgroup[progress= 60]{\footnotesize Pre-Study}{1}{4} \\
  \ganttbar{Literature review}{1}{4} \\
  \ganttbar{Small implementations}{2}{4} \\
    \ganttgroup[progress =0,group incomplete/.append style={fill=cyan}]{\footnotesize \color{teal} TW: Introduction \& methods }{1}{8} \\
  \ganttgroup[progress =0]{\footnotesize Simplified classifier }{5}{9} \\
  \ganttbar[name= initialDB]{Initial dataset extraction}{5}{7} \\
  \ganttmilestone[name= MSbinary]{ \color{red} Binary classifier}{7}{8} \ganttnewline
   \ganttbar[name= BinEval]{Testing and evaluating}{8}{9} \\
    \ganttgroup[progress =0,group incomplete/.append style={fill=cyan}]{\footnotesize \color{teal} TW: modifications \& review }{8}{15} \\
  \ganttgroup[progress =0]{\footnotesize Transport Hub Classifier}{10}{17} \\
  \ganttbar[name = datasetExt]{Generalize Dataset extraction}{10}{13}\\
  \ganttmilestone[name= MSclassifier]{Scaling up the classifier}{13}{14} \\
   \ganttbar[name = TrHubClass]{Testing and Evaluating}{13}{17}\\
  \ganttgroup[progress =0]{\footnotesize Vehicle Usage Prediction}{15}{19} \\
  \ganttbar[name = predmodel]{Create Predictive Model}{15}{17}\\
  \ganttmilestone[name= PredModelTrain]{Training the model}{17}{18} \\
   \ganttbar[name = PredModelEval]{Testing and Evaluating}{18}{19}\\  
     \ganttgroup[progress =0,group incomplete/.append style={fill=cyan}]{\footnotesize \color{teal} TW: Finalizing report \& defense}{15}{20} \\
     
        \ganttbar[]{TW: Discussion \& conclusion}{15}{18}\\  
		   \ganttbar[]{Presentation making}{16}{18}\\  
        \ganttbar[]{Prepare for defense}{18}{20}\\  
        
         \ganttgroup[progress label text = {},progress =0,group incomplete/.append style={fill=orange}]{\footnotesize \color{orange} Thesis defense}{21}{22}\\
         
         
  \ganttlink[link type =s-s]{elem1}{elem2}
  \ganttlink[link type =f-s]{initialDB}{MSbinary}
    \ganttlink[link type =f-s]{MSbinary}{BinEval}
 \ganttlink[link type =f-s]{datasetExt}{MSclassifier}
  \ganttlink[link type =f-s]{MSclassifier}{TrHubClass}
\ganttlink[link type =f-s]{predmodel}{PredModelTrain}
  \ganttlink[link type =f-s]{PredModelTrain}{PredModelEval}

  \label{thesiscal}
\end{ganttchart}
\end{flushleft}


%% SWOT ANALYSIS
\section{SWOT Analysis}
SWOT analysis is presented in the following table to demonstrate different aspects of the project and its learning outcome as shown in the following table. \\

\begin{tcbraster}[raster columns=3, raster equal height, raster column skip=0mm, raster row skip=0mm, 
]
\begin{tcolorbox}[blankest]\end{tcolorbox}
\begin{tcolorbox}[mybox={S}{helpful!50!internal}]
\begin{myitemize} % S
\item Familiarity with TensorFlow
\item Good overview of related literature

\end{myitemize}
\end{tcolorbox} 
\begin{tcolorbox}[mybox={W}{harmful!50!internal}]
\begin{myitemize} %W
\item Lack of knowledge in cluster computing
\item Lack of knowledge in Scala 

\end{myitemize}
\end{tcolorbox}
\begin{tcolorbox}[mybox={O}{helpful!50!external}]
\begin{myitemize} %O
\item Powerful computational resources
\item Good external supervision 

\end{myitemize}
\end{tcolorbox}
\begin{tcolorbox}[mybox={}{SO}]
\begin{myitemize} %SO
\item Apply deep learning on big data
\item Learn new skills e.g. Scala

\end{myitemize}
\end{tcolorbox}
\begin{tcolorbox}[mybox={}{WO}]
\begin{myitemize}
\item Hands-on cluster computing
\item Learn new skills e.g. Scala

\end{myitemize}
\end{tcolorbox}
\begin{tcolorbox}[mybox={T}{harmful!50!external}]
\begin{myitemize}
\item Irrelevance of dataset
\item Unreliable ground truth.
\end{myitemize}
\end{tcolorbox}
\begin{tcolorbox}[mybox={}{ST}]
\begin{myitemize}
\item Extract more relevant features from the current data.
\item Come up with good dataset and consistent ground truth.
\end{myitemize}
\end{tcolorbox}
\begin{tcolorbox}[mybox={}{WT}]
\begin{myitemize}
\item Extracting poor dataset for training the classifier.

\end{myitemize}
\end{tcolorbox}

\end{tcbraster}
  \addvspace{2cm}

\section{Used applications} % USED Application
Scania's \textbf{HopsWorks} clustered framework will be used throughout the project. Following is a list of tools and frameworks to work with:
\begin{itemize}
\item \textbf{TensorFlow:} Open-source python library for machine learning.
\item \textbf{Zeppelin:} Web-based notebook within the HopsWorks framework that supports coding in Scala, python and other programming languages.
\item \textbf{Apache Spark:} an open-source cluster computing that allows dealing with big data efficiently and manipulating databases.
\item \textbf{Apache Kafka:}  an open-source stream processing platform that helps in creating pipelines for real-time applications.
\item \textbf{SQL Databases:}  Knowledge of SQL is essential in the project in order to manipulate and extract desired information.

\end{itemize}
\end{document}

%%% End document
\end{document}